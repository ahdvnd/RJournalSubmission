% !TeX root = RJwrapper.tex
\title{A package for Cleaning and Analyzing Coursera OnDemand Data}
\author{by Aboozar Hadavand, Jeffrey Leek}

\maketitle

\abstract{%
An abstract of less than 150 words.
}

\subsection{Introduction}\label{introduction}

Why this paper?

\begin{itemize}
    \item talk about moocs
    \item one challenge in studying moocs is data
    \item even when data are available, the difficulty of data analysis makes it hard for researcher
    \item therefore, I provided this paper
    \item moocs have become massive laboratory for pedagogy
    \item in this paper I introduce the package and run an analysis of students progress using the package.
    \item statistics on coursera
\end{itemize}

It is hard to pin down the time of the birth of the first Massive Open
Online Course
(MOOC).\footnote{Some have claimed Sesame Street as the first MOOC. Delaney Parrish, "Sesame Street was the original MOOC," *BROOKINGS NOW*, The Brookings Institution, June 18, 2015, https://www.brookings.edu/blog/brookings-now/2015/06/18/sesame-street-was-the-original-mooc/}
But since the advent of more focused MOOCs pioneered by universities and
platforms such as Coursera, Udacity, and edX, reserachers have tried to
focus on studying MOOCs. There are fundamental differences between
traditional education and MOOCs was large enough to attract reserachers
to study students' behavior and outcomes. These differences are best
reflected in the definition of MOOCs by \cite{mcauley2010mooc} that
``{[}a{]}n online course with the option of free and open registration,
a publicly shared curriculum, and open-ended outcomes which integrates
social networking, accessible online resources \ldots{} and most
significantly builds on the engagement of learners who self-organize
their participation according to learning goals, prior knowledge and
skills, and common interests.''

Research on MOOCs few years with more data being accumulated and
collected. \cite{bozkurt2017trends} studied literature published on
MOOCs throught 2015 and found that the number of articles published on
the subject increased from 1 in 2008 to 170 in 2015. More research in
needed to fully understand the effectiveness, reach, limits, and the
potential of MOOCs. However, one of the main challenges in studying
MOOCs remains to be data. Data is not usually publically available since
it is owened by private MOOC providers and there are concerns about
privacy of students. More importantly, as \cite{lopez2017google} point
out, the size and complexity of MOOC data is an overwhelming challenge
to many researchers. Therefore, it is imperative to provide tools that
pave the way for more research on the new subject of MOOCs.

This paper introduces a package called \emph{crsra} based on the
statistical sofware R to help clean and analyze large loads of data from
the Coursera MOOCs. The advantages of the package are as follows: a)
faster loading of data for analysis, b) efficient method for combining
data from multiple courses and even across
institutions,\footnote{This is important since although MOOC researchers have access to thousands of students in their sample, few studies benefit from data across multiple courses and institutions. Such analysis helps draw more robust conclusions about student behaviors \citep{reich2015rebooting}.}
and c) provision of a set of functions for analysing student behaviors.

\subsection{Coursera On-Demand Data}\label{coursera-on-demand-data}

Coursera is one of the main providers of MOOCs that launched in January
2012. In fact, with over 25 million learners, Coursera is the biggest
provider in the world being followed by EdX, the MOOC provider that was
a result of a collaboration between Harvard Universit and MIT, with over
10 million users. Coursera has over 150 uiveristy partners from 29
countries and offers a toatl of 2000+ courses from computer science to
philosophy \citep{coursera}. In addition, Coursera offers 180+
specialization, Coursera's own credential system, and 4 fully online
Masters degrees. Courses include recorded video lectures, graded
assignment, quizzes, and discussion forums.

when coursera on-demand data started. history of coursera sharing their
data. different phases. they provide it to partners write about data
from coursera's gitbook what kind of demographic size of data examples
of hopkins courses provide a list of tables that connect different
students.

Talk about the advantage of not having relational databses

\subsection{crsra Package}\label{crsra-package}

talk about the package

\subsection{Analysis of student behavior on
Coursera}\label{analysis-of-student-behavior-on-coursera}

provide the analaysis here

\subsection{Discussion}\label{discussion}

\address{%
Aboozar Hadavand\\
Bloomberg School of Public Health, Johns Hopkins University\\
615 N. Wolfe Street\\ Baltimore, MD 21205, USA\\
}
\href{mailto:hadavand@jhu.edu}{\nolinkurl{hadavand@jhu.edu}}

\address{%
Jeffrey Leek\\
Bloomberg School of Public Health, Johns Hopkins University\\
615 N. Wolfe Street\\ Baltimore, MD 21205, USA\\
}
\href{mailto:jtleek@jhu.edu}{\nolinkurl{jtleek@jhu.edu}}

